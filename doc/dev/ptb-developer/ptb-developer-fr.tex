% Documentation francaise de l'API utilis�e pour r�aliser Plee the Bear
%
\documentclass{article}
\usepackage{comment}
\usepackage[Algorithme]{algorithm}
\usepackage{algorithmic}      % �criture d'algorithmes
\usepackage[francais]{babel}  % en fran�ais
\usepackage[latin1]{inputenc}
\usepackage[T1]{fontenc}
\usepackage{amssymb}          % symboles math�mathiques
\usepackage{amsmath}          % environnements math�mathiques
\usepackage{graphicx}         % insertion d'images
\usepackage{makeidx}          % index

\title{Plee the Bear - Guide du d�velopeur v1}
\author{Julien Jorge}
\date{\today}

\makeindex

\begin{document}

\maketitle
\newpage
\tableofcontents
\listoftables
\listoffigures
\listofalgorithms
\newpage

\section{Changements par rapport � la version pr�c�dente}
\begin{itemize}
\item Mercredi 14 septembre 2005 : Version 0 : version initiale.
\item Lundi 20 f�vrer : Version 1 : 
      \begin{itemize}
      \item s�paration du fichier en plusieurs petits fichiers ;
      \item mise � jour du contenu avec les nouvelles classes, et correction
            des informations obsol�tes ;
      \item ajout de l'index.
      \end{itemize}
\end{itemize}

%_______________________________________________________________________________
\section{Introduction}
L'objectif de ce document est de pr�senter l'ensemble des classes et
m�thodes utilis�es pour le jeu \emph{Plee the Bear} (PTB). Le sc�nario
est s�rement d�crit dans un document joint.

Toutes ces classes utilisent intensivement la classe \verb|claw::logger|, qui
devra donc �tre initialis�e au d�but du programme. On impl�mentera le programme
principal � l'aide de la classe \verb|claw::application|, ce qui nous
simplifiera la t�che.

%_______________________________________________________________________________
\section{Rendu visuel, images et graphisme}

\subsection{L'espace de noms visual \index{visual!espace de noms} }
Tout ce qui se rapporte au rendu visuel est dans l'espace de noms
\texttt{visual}. L'environnement doit �tre initialis� en �crivant :
\begin{verbatim}
    visual::screen::initialize();
\end{verbatim}
puis ferm� en �crivant :
\begin{verbatim}
    visual::screen::release();
\end{verbatim}

\subsubsection{Les images \index{visual!images} }
Les images sont repr�sent�es en m�moire par la classe
\verb|visual::image|. Elles peuvent �tre construites � partir d'un
objet \verb|claw::graphic::image|. On pr�f�rera utiliser des images
\emph{targa}, qui permettent de stocker une couche alpha pour la
transparence. Les dimensions des images doivent �tre une puissance
de~2. Vous pouvez charger une image \emph{targa} via
\verb|claw::graphic::targa|.

\subsubsection{Les objets graphiques}
Il y a deux types d'objets graphiques : les sprites et
les animations.

\subsubsection{Les sprites \index{visual!sprite} }
Les sprites sont des �l�ments affichables. Ce sont d'ailleurs les
seuls �l�ments r�ellement affichables. Les sprites sont d�finis par
une image et un rectangle englobant (ie. la partie de l'image qui nous
int�resse). Il est possible de lui appliquer divers effets : inversion
horizontale, verticale et transparence globale. Voyez la classe
\verb|visual::sprite| pour avoir plus de d�tails.

\subsubsection{Les animations \index{visual!animation} }
La classe \verb|visual::animation| est en quelque sorte une extension
de la classe \verb|visual::sprite| permettant de faire, justement, des
animations (s�ries de sprites r�gul�es par un framerate). Les
animations peuvent �tre lues un nombre fini ou infini de fois et
peuvent effectuer une lecture inverse lorsque la derni�re image est
atteinte.

\subsubsection{Les effets \index{visual!effet} }
Il est possible d'appliquer des effets visuels sur l'ensemble de
l'�cran et de mani�re plus g�n�rale sur n'importe quelle image. Tous
ces effets h�ritent de la classe \verb|visual::screen_effect|. Le nom
porte � confusion : ces filtres peuvent bel et bien �tre appliqu�s �
n'importe quelle image. Ils sont ind�pendants de l'existence de la
classe \verb|visual::screen|. Les effets dont le comportement �volue
dans le temps h�ritent de la classe
\verb|visual::progressive_screen_effect|. Le tableau
\ref{tab_effets_visuels} pr�sente les diff�rents effets impl�ment�s.

\begin{table}[!htb]
  \begin{center}
  \begin{tabular}{|c|l|}
    \hline
    Effet & Description \\
    \hline
    blur\_effect & Un effet de flou l�ger. \\
    fade\_effect & Un effet de fondu en noir. \\
    grey\_effect & Met l'image en noir et blanc. \\
    negative\_effect & Inverse les couleurs de l'image. \\
    pixel\_effect & Effet de pixellisation. \\
    quake\_effect & << Secoue >> l'image pour donner une impression de
                    tremblement. \\
    zoom\_effect & Un effet de flou de vitesse vers l'int�rieur de l'�cran. \\
    \hline
  \end{tabular}
  \caption{Effets visuels envisag�s}
  \label{tab_effets_visuels}
  \end{center}
\end{table}

\subsubsection{L'�cran \index{visual!ecran@�cran} }
L'�cran est symboliquement repr�sent� par la classe
\verb|visual::screen|. C'est sur l'�cran que se dessinent les
sprites et que s'appliquent les effets visuels. Les sprites y sont
dessin�s via la m�thode \texttt{render} et les effets y sont appliqu�s
via la m�thode \texttt{apply\_effect}. Pour initialiser une nouvelle
session de rendu, appelez \texttt{begin\_render}. Pour initialiser une
nouvelle session d'effets, appelez \texttt{begin\_effect}. Tout appel �
\texttt{begin\_*} doit se terminer par un appel � \texttt{end\_*}. Il
est possible d'appliquer des effets pendant le
rendu. L'algorithme~\ref{algo_ex_visual} pr�sente un exemple
d'utilisation de la classe \verb|visual::screen|. Il est possible de
sauver l'�cran dans un fichier bitmap en appelant la m�thode
\texttt{shot}. 

\begin{algorithm}[!htb]
  \begin{center}
  \begin{algorithmic}
    \STATE\COMMENT{Two lists of renderable items}
    \STATE{var background : list of pair<renderable, coordinate\_2d>}
    \STATE{var foreground : list of pair<renderable, coordinate\_2d>}
    \STATE
    \STATE\COMMENT{Begining of the rendering process}
    \STATE{screen.begin\_render()}
    \STATE
    \STATE\COMMENT{Draw the background}
    \FORALL{item in background}
      \STATE{screen.render(item.second, item.first)}
    \ENDFOR
    \STATE
    \STATE\COMMENT{Application of an effect}
    \STATE{screen.begin\_effect()}
    \STATE{screen.apply\_effect( blur\_effect() ) }
    \STATE{screen.end\_effect()}
    \STATE
    \STATE\COMMENT{Draw the foreground}
    \FORALL{item in foreground}
      \STATE{screen.render(item.second, item.first)}
    \ENDFOR
    \STATE{screen.end\_render()}
  \end{algorithmic}
  \caption{Exemple d'utilisation des classes de l'espace de nom
           \texttt{visual}}
  \label{algo_ex_visual}
  \end{center}
\end{algorithm}

\subsubsection{Le gestionnaire d'images \index{visual!gestionnaire d'images} }
La classe \verb|visual::image_manager| permet de g�rer des ressources
graphiques. Elle propose une m�thode d'ajout d'image et une m�thode de
r�cup�ration d'image. Chaque image a un nom unique au sein du gestionnaire.

\subsection{Gestion des ressources graphiques \index{ressource!gestionnaire
            d'image} }
Au sein du jeu, on distingue deux types d'images : les images locales � un
niveau et les images pr�sentes dans tout le jeu (joueur, bonii, etc.). Pour
simplifier l'utilisation de ces images, on utilise le singleton
\verb|ptb::image_manager|. Celui-ci propose deux m�thodes d'ajouts, selon que
l'image soit globale ou pas, mais une seule m�thode d'acc�s, masquant ainsi la
nature de l'image.

\subsection{\'El�ments visuels du jeu }
Dans le jeu, les objets souhaitant pouvoir �tre affich�s doivent h�riter de la
classe \verb|ptb::renderable|
 \index{renderable \emph{(classe ptb::)}}
(plus pr�cis�ment de \verb|ptb::renderable_item|
 \index{renderable item \emph{(classe ptb::)}}
ou de \verb|ptb::renderable_messageable|
 \index{renderable messageable \emph{(classe ptb::)}}
).
%_______________________________________________________________________________
\section{Rendu sonore, sons et environnement}

\index{son|see{audio}}
\index{musique|see{audio}}

\subsection{L'espace de noms audio \index{audio!espace de noms} }

Tout ce qui se rapporte au rendu sonore est dans l'espace de noms
\texttt{audio}. L'environnement doit �tre initialis� en �crivant :
\begin{verbatim}
    audio::sound::initialize();
\end{verbatim}
puis ferm� en �crivant :
\begin{verbatim}
    audio::sound::release();
\end{verbatim}

\subsubsection{Les sons \index{audio!son} }
Les sons sont repr�sent�s par la classe \verb|audio::sound|. Elle poss�de quatre
m�thodes de lecture, permettant de contr�ler l'effect appliqu� et l'origine du
son. Un son peut �tre jou� plusieurs fois simultan�ment. Cependant, il n'y a
presque aucun contr�le lors d'un arr�t explicite du son. Un appel �
\texttt{stop()} arr�te la plus ancienne lecture. Les sons doivent �tre associ�s
� un gestionnaire de sons lors de leur construction. Les fichiers utilis�s sont
de type wave.

\subsubsection{Les musiques \index{audio!musique} }
Les musiques sont repr�sent�es par la class \verb|audio::music|. Il ne peut y
avoir qu'une seule musique jou�e � un instant donn�. Les fichiers utilis�s
peuvent �tre de n'importe quel type audio, on pr�f�rera cependant le format
\emph{Ogg Vorbis}.

\subsubsection{Les effets \index{audio!effet} }
La classe \verb|audio::sound_effect| offre un certain contr�le sur la fa�on dont
sont jou�s les sons. Pour l'instant, les seuls param�tres contr�lables sont
le volume et le nombre de lectures.

\subsection{Gestion des ressources sonores \index{ressource!gestionnaire de
            sons} \index{ressource!musique} }
De la m�me mani�re que pour les images, on distingue deux types de sons et de
musiques : ceux locaux � un niveau et ceux
pr�sents dans tout le jeu. Pour simplifier l'utilisation de ces ressources, on
utilise le singleton \verb|ptb::sound_manager|.
Celui-ci propose deux m�thodes d'ajouts pour chaque ressource, selon qu'elle
soit globale ou pas, mais une seule m�thode d'acc�s, masquant ainsi la
globalit� de la ressource.

De plus, le gestionnaire de sons a une m�thode permettant de positionner les
<<~oreilles~>> du joueur dans le niveau, afin d'avoir un effet d'estompement
\index{audio!effet!estompement} des sons en fonction de la distance du joueur �
la source.

%_______________________________________________________________________________
\section{Classes conceptuelles \index{concept!espace de noms} }

\subsection{L'espace de noms concept \index{concept!espace de
            noms} }
Toutes les classes g�n�riques sont dans l'espace de noms \texttt{concept}.

\subsubsection{Les classes contenant une collection d'objets
               \index{concept!item container \emph{(classe)}} }
Il y a dans ce jeu des classes contenant une collection d'objets et
appliquant de temps en temps la m�me m�thode � tous ces
objets. En g�n�ral ce genre de classe a pour seul but d'effectuer une
t�che comme :
  \begin{center}
  \begin{algorithmic}
    \FORALL{item in my\_collection}
    \STATE{item->do\_something()}
    \ENDFOR
  \end{algorithmic}
  \end{center}
Le probl�me est qu'il se peut que la m�thode \texttt{do\_something}
supprime un des objets contenus et poser des probl�mes dans le parcours
de la liste.

Pour �viter ce genre de probl�mes, la classe
\verb|concept::item_container| propose une solution. Cette classe
garantie � toute h�riti�re qu'elle n'aura pas d'ajout ou de
suppression d'item pendant son traitement, sous les conditions
suivantes :

\begin{itemize}
\item la classe contenante doit appeler sa m�thode \texttt{lock} au
      d�but du traitement et la m�thode \texttt{unlock} � la fin ;
\item les items s'annoncent � la classe contenante via la m�thode
      \texttt{register\_item} et annoncent leur d�part via la m�thode
      \texttt{release\_item} ;
\end{itemize}

Les items ajout�s ou retir�s pendant le traitement de la classe
contenante ne sont alors effectivement ajout�s ou retir�s que lors de
l'appel � la m�thode \texttt{unlock}.
\paragraph{Attention :} aucun des items contenus ne devrait �tre supprim� de la
m�moire (appel � \texttt{delete}) pendant la boucle.

\subsubsection{Les classes positionnant des objets dans une table � deux
               dimensions \index{concept!static map \emph{(classe)}} }
\label{section-concept-static_map}

Il faut quelquefois positioner un petit nombre d'objets (petit par rapport � la
taille de l'espace) dans un grand espace en deux dimensions. Par exemple, il
faut placer les d�corations, les murs dans une table de la taille du niveau.
Bien s�r, la plupart des positions dans la table ne contiennent pas d'objets
et occupent de la place inutilement. Pour r�duire la quantit� de m�moire
utilis�e, on peut faire appel � la classe \verb|concept::static_map|. Voici ce
que dit la documentation :

\begin{quote}
La table est divis�e en petites bo�tes dans lesquelles les items sont list�s.
Disons que l'on aie une table de $1~000 \times 1~000$ cellules. Il n'y a pas des
objets dans chaque cellule, mais chacune d'elles prend de la place en m�moire.
En utilisant une \verb|static_map| de taille $100 \times 100$ (avec les bo�tes
de taille $10 \times 10$), on peut r�duire la m�moire utilis�e par un carr� de
$100$ cellules vides � celle utilis�e par une seule cellule. S'il y a des items
dans une bo�te, on les liste dans une cellule ; la m�moire utilis�e est la m�me
mais l'acc�s est un petit peu plus long.
\end{quote}

En plus du type des items, cette classe a besoin d'un param�tre de traits sur
ce type d'item. La seule m�thode n�cessaire est \verb|get_bounding_box()|, qui
renvoie une bo�te englobante d'un objet.

\paragraph{Attention :} les items plac�s dans la table ne peuvent ni bouger ni 
�tre supprim�s.

%_______________________________________________________________________________
\section{La communication entre les objets \index{communication!espace de 
         noms} }

\subsection{L'espace de noms communication \index{communication!espace de
            noms} }
Tout ce qui se rapporte � la communication entre objets est dans l'espace de
noms \texttt{communication}.

\subsubsection{Le bureau de poste \index{communication!post office
                \emph{(classe)}} }
La classe \verb|communication::post_office| agit comme un bureau de poste.
Entendez par l� quelle permet de faire communiquer les items qui en d�pendent.
Vous pouvez th�oriquement avoir autant de bureaux de poste que vous le
souhaitez. Tous les objets d�pendants d'un bureau de poste doivent avoir un nom
unique. Cependant, deux items d�pendant de bureaux de poste diff�rents peuvent
avoir le m�me nom.

Il y a deux mani�res d'envoyer un message. La m�thode \verb|post_message()|
ajoute le message dans la queue des messages du destinataire ; alors que la
m�thode \verb|send_message()| permet de faire traiter le message par le
destinataire. La premi�re m�thode garde une r�f�rence vers le message envoy�,
il ne faut donc pas que celui ci soit d�truit avant d'�tre trait�. Il est � la
charge de l'�metteur de d�truire les messages envoy�s.

Appelez la m�thode \verb|process_messages()| pour traiter les messages en
attente dans la file d'attente des items.

Les objets souhaitant pouvoir recevoir des messages doivent h�riter de la
classe \verb|communication::messageable| \index{communication!messageable} et
s'enregistrer aupr�s d'un bureau de poste.

\subsubsection{Les messages \index{communication!message \emph{(classe)}} }
Les messages transmis doivent h�riter de la classe
\verb|communication::message|. Chaque type de message doit avoir un code
identifiant son contenu (afin de faciliter le traitement par les destinataires).

\subsection{Organisation de la communication dans le jeu
            \index{communication!message \emph{(classe)}} }
Dans le jeu, il n'y a qu'un seul bureau de poste, repr�sent� par le singleton
\verb|ptb::post_office|
 \index{post office \emph{(classe ptb::)}}
 \index{bureau de poste}.

Les items souhaitant recevoir des messages doivent h�riter de
\verb|ptb::messageable_item|
 \index{messageable item \emph{(classe ptb::)}}
ou de \verb|ptb::renderable_messageable|
 \index{renderable messageable \emph{(classe ptb::)}}
.

%_______________________________________________________________________________
\section{Entr�es, clavier et joystick}

\index{entrees@entr�es|see{input}}

\subsection{L'espace de noms input \index{input!espace de noms} }
Tout ce qui se rapporte aux entr�es est dans l'espace de noms \texttt{input}.
L'environnement doit �tre initialis� en �crivant :
\begin{verbatim}
    input::system::initialize();
\end{verbatim}
puis ferm� en �crivant :
\begin{verbatim}
    input::system::release();
\end{verbatim}

\subsubsection{Le syst�me d'entr�es \index{input!systeme@syst�me} }
Le syst�me permettant de g�rer le clavier et les joysticks est implant� par la
classe \verb|input::system|. Il propose une m�thode pour cr�er un nouveau
contr�leur (clavier ou joystick) et une pour d�truire un contr�leur quelconque.
Vous devez utiliser la m�me instance de \verb|input::system| pour cr�er et
d�truire une instance d'un contr�leur. les contr�leurs associ�s � une instance
de \verb|input::system| deviennent invalides si l'instance est d�truite.

La m�thode \texttt{refresh()} met � jour l'�tat de tous les contr�leurs,
\textbf{quelle que soit l'instance} de
\verb|input::system| ayant servi � le cr�er. Cependant, vous devrier consid�rer
que ce n'est pas le cas et appeler \texttt{refresh()} sur toutes les instances
de \verb|input::system|.

Cette contrainte est d�e � SDL \emph{(Simple Directmedia Layer)}. SDL
g�re les entr�es sous la forme d'�v�nements et il n'est pas possible de ne 
r�cup�rer que les �v�nements d'un contr�leur en particulier.

\subsubsection{Le clavier \index{input!clavier} }
Le clavier est repr�sent� par la classe \verb|input::keyboard|. Il n'y a aucun
moyen de choisir quel est le clavier � associer � l'instance, tant pis pour ceux
qui ont plusieurs claviers branch�s � leur PC.

Les touches sont identifi�es de mani�re unique par une valeur de type
\verb|input::keyboard::key_code|. La classe propose le couple de m�thodes
\texttt{begin()}/\texttt{end()} permettant d'it�rer sur les touches press�es et
une m�thode \texttt{empty()} indiquant s'il n'y a pas de touches press�es.

La classe propose quatre m�thodes pour manipuler les codes de touches. La
m�thode \texttt{is\_symbol()} indique si un code correspond � un caract�re
imprimable ; \texttt{code\_to\_char()} convertit un de ces caract�res imprimable
en \texttt{char}. Les m�thodes \texttt{get\_name\_of()} et
\texttt{set\_name\_of()} permettent respectivement d'obtenir et de d�finir le
nom associ� � un code de touche.

\paragraph{Note :}la m�thode \texttt{refresh()} ne doit �tre utilis�e que par
la classe \verb|input::system|.

\subsubsection{Le joystick \index{input!joystick} \index{input!manette}}
Les joysticks et manettes sont repr�sent�s par la classe \verb|input::joystick|.
Le constructeur prend en param�tre l'identifiant du contr�leur associ�. Un
identifiant est une valeur positive strictement inf�rieure au nombre de 
joysticks branch�s. La m�thode statique
\verb|input::joystick::number_of_joysticks()| permet d'obtenir le nombre de
joysticks branch�s.

Les boutons sont identifi�es de mani�re unique par une valeur de type
\verb|input::joystick::joy_code|. La classe propose le couple de m�thodes
\texttt{begin()}/\texttt{end()} permettant d'it�rer sur les boutons press�es et
une m�thode \texttt{empty()} indiquant s'il n'y a pas de boutons press�es. La
croix directionnelle correspond � huit codes, un pour chaque direction.

La classe propose deux m�thodes pour manipuler les codes des boutons. Les
m�thodes \texttt{get\_name\_of()} et \texttt{set\_name\_of()} permettent
respectivement d'obtenir et de d�finir le nom associ� au code d'un bouton.

\paragraph{Note :}la m�thode \texttt{refresh()} ne doit �tre utilis�e que par
la classe \verb|input::system|.


%_______________________________________________________________________________
\section{Objets, lois de physiques et interactions}

\subsection{L'espace de noms univers \index{universe!espace de noms} }
Tout ce qui se rapporte � l'univers (entit�s de base, lois de la physique, etc.)
est dans l'espace de noms \texttt{universe}.

\subsubsection{Qu'est-ce qu'un objet physique ? \index{universe!objet
               physique} }
Un objet physique (classe \verb|universe::physical_item|) est un objet
sur lequel s'appliquent les lois de la physique ; un objet qui peut
int�ragir, rentrer en contact avec ses semblables ; un objet qui subit
la gravit�, a une vitesse, une acc�leration, etc. Les collisions sont
d�tect�es avec la m�thode des boites englobantes.

Lors d'une collision, les objets en contacts sont inform�s par un appel � leur
m�thode \texttt{collision}. Le syst�me leur donne alors une r�f�rence de l'autre
item et les positions de chaque objets. Cette m�thode n'a pas le droit de
modifier directement la position des objets (m�thode \texttt{set\_position()}),
elle peut cependant modifier les param�tres de positions qui lui sont pass�s.

Les objets du jeu h�riteront de \verb|universe::base_entity|, qui propose une
m�thode \texttt{progress()} sens�e effectuer l'�volution de l'objet et h�rite de
\verb|universe::physical_item|. 

\subsubsection{Quelles sont les lois physiques ? \index{universe!lois de la
               physique} }
La classe \verb|universe::physic_rules| repr�sente, autant qu'elle le
peut, le comportement physique des objets du monde. Elle calcule la
nouvelle position de tous les objets mobiles et d�tecte les
collisions. Son utilisation se fait comme pr�sent� dans
l'algorithme~\ref{algo_ex_physic_rules}.

\begin{algorithm}[!htb]
  \begin{center}
  \begin{algorithmic}
    \STATE{physic\_rules pr}
    \STATE\COMMENT{Tell the physic which items are interesting}
    \STATE{pr.begin\_listing()}
    \FORALL{item in the active area}
      \STATE{pr.add\_item(item)}
    \ENDFOR
    \STATE{pr.end\_listing()}
    \STATE\COMMENT{Apply the rules}
    \STATE{pr.progress()}
  \end{algorithmic}
  \caption{Exemple d'utilisation de la classe universe::physic\_rules }
  \label{algo_ex_physic_rules}
  \end{center}
\end{algorithm}

\subsubsection{Les mouvements forc�s \index{universe!mouvements
               forces@mouvements forc�s} }
Il est possible d'imposer un mouvement � un objet. Dans ce cas, cet
objet ne subira plus les forces de la physique. L'application de ces
mouvements est effectu�e lors de l'appel � \verb|physic_rules::progress|.
Il n'est donc plus n�cessaire de s'occuper d'un mouvement forc� une fois que
celui-ci a �t� appliqu� � un objet.  Le tableau~ \ref{tab_mouvements_forces}
pr�sente les diff�rents mouvements envisag�s. Lorsque le mouvement fait
intervenir deux objets, celui auquel le mouvement est appliqu� est l'objet $A$ ;
l'objet $B$ �tant en g�n�ral un point de r�f�rence.


\begin{table}[!htb]
  \begin{center}
  \begin{tabular}{|c|l|}
    \hline
    Mouvement & Description \\
    \hline
    forced\_join & L'objet $A$ va rejoindre l'objet $B$ \\
    forced\_rotation & Fait tourner l'objet $A$ autour de l'objet $B$
    \\
    forced\_tracking & L'objet $A$ reste � une distance
                       constante de l'objet $B$ \\
    forced\_train & Fait suivre un chemin � l'objet $A$ \\
    \hline
  \end{tabular}
  \caption{Mouvements forc�s envisag�s}
  \label{tab_mouvements_forces}
  \end{center}
\end{table}

\subsubsection{Structure du monde \index{universe!monde} }
Le monde est repr�sent� par la classe \verb|universe::world|. C'est un
conteneur de \verb|universe::physical_item| (pour les murs et tous ces
objets immobiles et � dur�e de vie infinie) et de \verb|universe::base_entity|
(pour tous les objets <<~vivants~>>). De plus le monde poss�de des lois
physiques.

Le monde sans ses entit�s est plut�t vide (il ne reste que les
murs). Vide dans le sens o� il y a beaucoup de coordonn�es sans
murs. On utilise donc une instance de \verb|concept::static_map()| (voir
\ref{section-concept-static_map}) pour les stocker.

\subsection{Le monde dans le jeu \index{monde} }
Dans le jeu, le monde est repr�sent� par la classe \verb|ptb::world|, qui
h�rite de \verb|universe::world|. La seule m�thode ajout�e est
\texttt{render()}, qui permet d'afficher les objets stock�s.
%_______________________________________________________________________________
\section{Manipulation de texte}

\subsection{L'espace de noms text \index{text!espace de noms} }
Tout ce qui se rapporte au traitement de textes est dans l'espace de noms
\texttt{text}.

\subsection{Polices de caract�res \index{text!police} }
Les polices de caract�res sont repr�sent�es par la classe \verb|text::font|. Il
faut passer l'image des caract�res au constructeur, ainsi que la taille des
caract�res. Toutes les polices sont � chasse fixe.

L'image doit contenir, dans l'ordre : les minuscules, les majuscules,
les chiffres, l'espace et les caract�res sp�ciaux :

\begin{verbatim}
! " # $ % & \ ( ) *
+ , - . / : ; < = >
? @ [ \ ] ^ _ ` { |
} ~ � � � � � � � �
� � � � � �
\end{verbatim}

\subsection{Mesures \index{text!mesure} }
La classe \verb|text::text_metric| permet de mesurer la taille d'un texte
associ� � une police, en nombre de caract�res ou de pixels.


%_______________________________________________________________________________
\section{Interface utilisateur}

%-------------------------------------------------------------------------------
\subsection{L'espace de nom gui \index{gui!espace de noms} }
Tout ce qui se rapporte � l'interface utilisateur est dans l'espace de noms
\texttt{gui}.

\subsection{Composant visuel de base
            \index{gui!composant visuel} }

Tous les composants visuels h�ritent de la classe \verb|gui::visual_component|.
La plupart des composants sont contenus dans un composant parent. C'est ce
composant qui se chargera de d�truire ses composants fils lors de sa
destruction. Un composant a une taille et une position relative au coin en
haut � gauche du composant parent. Les composants fils sont enti�rement inclus
(au sens g�om�trique) dans le composant parent (pas de d�bordement).

Chaque composant se charge d'afficher ses fils (c'est automatique). Les fils
peuvent red�finir la m�thode \texttt{display()} pour s'afficher. Il n'y a
aucune garantie sur l'ordre d'affichage des composants fils.

\subsection{Les fen�tres \index{gui!fenetre@fen�tre} }

La fen�tre est le premier composant graphique. Elle est repr�sent�e par la
classe \verb|gui::window|. Il s'agit d'un cadre avec un fond et un bord,
destin� � contenir des contr�les. Il faut indiquer au constructeur quels sont
les sprites � utiliser pour le fond, les bords et les coins. Les sprites de
fond et de bords sont r�p�t�s pour remplir l'espace.

Actuellement l'affichage n'affiche que les morceaux entiers de l'arri�re plan
(pas de coupure sur les bords).

\subsection{Zone de texte en lecture seule
            \index{gui!zone de texte} }

Il s'agit d'un composant simple, charg� d'afficher du texte � l'�cran. Il est
repr�sent� par la classe \verb|gui::static_text|. La taille peut �tre ajust�e
automatiquement au texte, autrement il sera coup� au niveau des espaces.

Il faut indiquer au contructeur quelle est la police � utiliser pour afficher
le texte. Cette police sera d�truite en m�me temps que le composant.

\subsection{Zone de texte en lecture seule et d�filable
            \index{gui!zone de texte defilable@zone de texte d�filable} }

Cette classe (\verb|gui::multi_page|) permet d'afficher un texte qui ne 
tiendrait pas � l'�cran en permettant de faire d�filer le texte.

\subsection{Zone de saisie \index{gui!zone de saisie} }

La classe \verb|gui::text_input| permet de r�cup�rer une ligne de texte entr�e
par l'utilisateur.


%_______________________________________________________________________________
\section{Les outils du jeu \index{outils} }

%-------------------------------------------------------------------------------
\subsection{Collisions \index{collision} }

\subsubsection{Prise de contr�le \index{controle@contr�le} }

Nous n'avons pas encore choidi comment d�terminer quel est l'objet qui prend le
contr�le lors d'une collision.

\subsubsection{Gestion �v�nementielle \index{collisions!evenement@�v�nement} }
Il est possible d'automatiser le comportement � adopter lors d'une collision �
l'aide des classes filles de la classe \verb|ptb::collision_event|. Les
�v�nements actuellement implant�s sont :
\begin{itemize}
\item \texttt{collision\_event\_align} : aligne l'autre item en utilisant
      l'alignement pass� comme param�tre template
      (voir~\ref{section-alignements}) ;
\item \texttt{collision\_event\_align\_accelerate} : aligne l'autre item de la
      m�me mani�re que \texttt{collision\_event\_align} et lui applique
      ensuite une acc�l�ration pour l'�loigner de l'item courant ;
\item \texttt{collision\_event\_align\_stop} : semblable �
      \texttt{collision\_event\_align}, avec ceci de plus que l'autre item est
      arr�t�.
\end{itemize}

La classe \verb|ptb::base_item| contient une table permettant d'affecter un
traitement de collision en fonction de l'origine de l'autre item.

\subsection{Alignements de rectangles \index{alignement} }
\label{section-alignements}
L'ensemble des classes filles de \verb|ptb::alignment| permet d'aligner deux
rectangles. Les param�tres n�cessaires sont la bo�te fixe, la bo�te � aligner et
l'ancienne position de la bo�te � aligner. L'alignement se fait de mani�re
naturelle.

\paragraph{Note :}l'ancienne position de l'autre item doit �tre diff�rente de 
la position actuelle pour les alignements en coin.

\subsubsection{Zones cr��es par un item \index{zone} }
En tra�ant des droites passant par chaque c�t� de la bo�te englobante d'un item,
on fait appara�tre neuf zones que l'on num�rote conventionnellement de la
gauche vers la droite puis du haut vers le bas ; la premi�re �tant 0 ou 1 selon
l'indi�age qui vous arrange.

%-------------------------------------------------------------------------------
\subsection{Objets de base \index{base} }
Les items de base du jeu sont au nombre de quatre : \verb|ptb::base_item|,
\index{base item \emph{(classe ptb::)} }
\verb|ptb::renderable_item|, \verb|ptb::messageable_item| et
\index{renderable item \emph{(classe ptb::)} }
\index{messageable item \emph{(classe ptb::)} }
\verb|ptb::renderable_messageable|. Les trois derniers h�ritent du premier.
\index{renderable messageable \emph{(classe ptb::)} }

Chaque instance d'une de ces classes a un identifiant unique pendant tout le
jeu. \index{identifiant}

La classe de base a une m�thode \texttt{is\_renderable()} qui renverra vrai si
l'arbre contenant la classe instanci�e a un noeud \verb|ptb::renderable|.

Il est possible de param�trer dynamiquement certains champs � l'aide des
m�thodes \texttt{set\_field()}. Les param�tres sont de type \texttt{int},
\texttt{float} et \verb|std::string|. La version \texttt{void*} va dispara�tre
au profit d'une version \verb|visual::sprite| et d'une autre
\verb|visual::animation|. La m�thode \texttt{is\_valid()} doit retourner vrai
uniquement si tous les param�tres dynamiques n�cessaires ont �t� configur�s.

La classe de base a une table de neuf \texttt{collision\_event} permettant
d'automatiser la gestion des collisions en fonction de la zone de provenance
de l'autre item. L'�v�nement par d�faut ne fait aucune action. Vous pouvez
changer l'�v�nement associ� � une zone avec la m�thode
\texttt{set\_collision\_event()}. Appelez la m�thode
\texttt{default\_collision\_event()} pour appliquer l'�v�nement qui convient.

La classe \verb|ptb::renderable_item| n'ajoute aucun champ ou m�thode et permet
juste d'indiquer que l'objet h�ritant est affichable.

La classe \verb|ptb::messageable_item| h�rite de
\verb|communication::messageable| \index{communication!messageable} et permet
donc aux classes filles de recevoir des messages.

La classe \verb|ptb::renderable_messageable| combine les propri�t�s des deux
classes pr�c�dentes.

%-------------------------------------------------------------------------------
\subsection{Population \index{population} }

La classe singleton \verb|ptb::population| stocke et g�re la cr�ation et la
suppression des entit�s. Une m�thode \texttt{exists()} permet de savoir s'il
existe un item avec un identifiant particulier.

%-------------------------------------------------------------------------------
\subsection{Entit�s partag�es \index{entite partagee@entit� partag�e} }

Il est quelquefois n�cessaire de stocker un pointeur vers un item que l'on ne
contr�le pas, c'est � dire que l'item peut cesser d'exister � n'importe quel
moment. La classe \verb|ptb::item_handle| permet de manipuler ce genre d'items
de mani�re s�re. Elle d�fini les op�rateurs de d�r�f�rence, permettant ainsi
de l'utiliser comme on utiliserait le pointeur qu'elle contient. La m�thode
\texttt{is\_valid()} permet de savoir si le pointeur contenu est toujours
valide.

%-------------------------------------------------------------------------------
\subsection{Calques \index{calque} \index{layer|see{calque}} }

Les calques h�ritent tous de la classe \verb|ptb::layer|. Nous en avons envisag�
quatre, mais seulement trois sont impl�ment�s.

\subsubsection{Calque de d�coration \index{calque!decoration@d�coration} }

Les calques de d�coration ne contiennent que des sprites et des animations. Les
deux sont stock�s dans un \verb|concept::static_map| pour �conomiser l'espace.
Bien entendu, les positions des d�corations sont fixes.

\subsubsection{Calque d'activit� \index{calque!activite@activit�} }

C'est dans le calque d'activit� que se passe toute l'action. Bien entendu il n'y
aura qu'un seul calque d'activit� dans la pile des calques. Ce calque contient
une instance de \verb|ptb::world| et sert juste de conteneur.

\subsubsection{Calque de motif \index{calque!motif} }

Le calque de motif n'est pas encore impl�ment�.

\subsubsection{Calque de status \index{calque!status} }
Le calque de status devrait �tre unique et au dessus de tous les autres calques.
Il ne sert qu'� afficher des informations sur l'�tat du (des) joueur(s) (points,
vies, �nergie et autres).

%-------------------------------------------------------------------------------
\subsection{Cam�ra \index{camera@cam�ra} }

Une cam�ra est repr�sent�e par la classe \verb|ptb::camera|. Il s'agit
simplement d'un rectangle plac� dans le monde. Il y a deux mani�res de
positionner une cam�ra : soit en indiquant la position du coin en haut � gauche,
soit en indiquant la position du centre. Dans les deux cas, la cam�ra est
ajust�e pour ne pas sortir du niveau. La m�thode \texttt{get\_focus()} permet
de r�cup�rer la position de la cam�ra.

%-------------------------------------------------------------------------------
\subsection{Niveaux \index{niveau} }

La classe \verb|ptb::level| repr�sente un niveau. Elle contient la cam�ra du
niveau (position � calculer en fonction des positions des joueurs) et les
calques.

Il est envisageable d'impl�menter des niveaux particuliers en h�ritant de cette
classe.


%_______________________________________________________________________________

\printindex

\end{document}
