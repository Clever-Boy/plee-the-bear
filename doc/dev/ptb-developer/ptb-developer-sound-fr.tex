%_______________________________________________________________________________
\section{Rendu sonore, sons et environnement}

\index{son|see{audio}}
\index{musique|see{audio}}

\subsection{L'espace de noms audio \index{audio!espace de noms} }

Tout ce qui se rapporte au rendu sonore est dans l'espace de noms
\texttt{audio}. L'environnement doit �tre initialis� en �crivant :
\begin{verbatim}
    audio::sound::initialize();
\end{verbatim}
puis ferm� en �crivant :
\begin{verbatim}
    audio::sound::release();
\end{verbatim}

\subsubsection{Les sons \index{audio!son} }
Les sons sont repr�sent�s par la classe \verb|audio::sound|. Elle poss�de quatre
m�thodes de lecture, permettant de contr�ler l'effect appliqu� et l'origine du
son. Un son peut �tre jou� plusieurs fois simultan�ment. Cependant, il n'y a
presque aucun contr�le lors d'un arr�t explicite du son. Un appel �
\texttt{stop()} arr�te la plus ancienne lecture. Les sons doivent �tre associ�s
� un gestionnaire de sons lors de leur construction. Les fichiers utilis�s sont
de type wave.

\subsubsection{Les musiques \index{audio!musique} }
Les musiques sont repr�sent�es par la class \verb|audio::music|. Il ne peut y
avoir qu'une seule musique jou�e � un instant donn�. Les fichiers utilis�s
peuvent �tre de n'importe quel type audio, on pr�f�rera cependant le format
\emph{Ogg Vorbis}.

\subsubsection{Les effets \index{audio!effet} }
La classe \verb|audio::sound_effect| offre un certain contr�le sur la fa�on dont
sont jou�s les sons. Pour l'instant, les seuls param�tres contr�lables sont
le volume et le nombre de lectures.

\subsection{Gestion des ressources sonores \index{ressource!gestionnaire de
            sons} \index{ressource!musique} }
De la m�me mani�re que pour les images, on distingue deux types de sons et de
musiques : ceux locaux � un niveau et ceux
pr�sents dans tout le jeu. Pour simplifier l'utilisation de ces ressources, on
utilise le singleton \verb|ptb::sound_manager|.
Celui-ci propose deux m�thodes d'ajouts pour chaque ressource, selon qu'elle
soit globale ou pas, mais une seule m�thode d'acc�s, masquant ainsi la
globalit� de la ressource.

De plus, le gestionnaire de sons a une m�thode permettant de positionner les
<<~oreilles~>> du joueur dans le niveau, afin d'avoir un effet d'estompement
\index{audio!effet!estompement} des sons en fonction de la distance du joueur �
la source.
