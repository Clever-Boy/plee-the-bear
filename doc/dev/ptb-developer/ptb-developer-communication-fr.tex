%_______________________________________________________________________________
\section{La communication entre les objets \index{communication!espace de 
         noms} }

\subsection{L'espace de noms communication \index{communication!espace de
            noms} }
Tout ce qui se rapporte � la communication entre objets est dans l'espace de
noms \texttt{communication}.

\subsubsection{Le bureau de poste \index{communication!post office
                \emph{(classe)}} }
La classe \verb|communication::post_office| agit comme un bureau de poste.
Entendez par l� quelle permet de faire communiquer les items qui en d�pendent.
Vous pouvez th�oriquement avoir autant de bureaux de poste que vous le
souhaitez. Tous les objets d�pendants d'un bureau de poste doivent avoir un nom
unique. Cependant, deux items d�pendant de bureaux de poste diff�rents peuvent
avoir le m�me nom.

Il y a deux mani�res d'envoyer un message. La m�thode \verb|post_message()|
ajoute le message dans la queue des messages du destinataire ; alors que la
m�thode \verb|send_message()| permet de faire traiter le message par le
destinataire. La premi�re m�thode garde une r�f�rence vers le message envoy�,
il ne faut donc pas que celui ci soit d�truit avant d'�tre trait�. Il est � la
charge de l'�metteur de d�truire les messages envoy�s.

Appelez la m�thode \verb|process_messages()| pour traiter les messages en
attente dans la file d'attente des items.

Les objets souhaitant pouvoir recevoir des messages doivent h�riter de la
classe \verb|communication::messageable| \index{communication!messageable} et
s'enregistrer aupr�s d'un bureau de poste.

\subsubsection{Les messages \index{communication!message \emph{(classe)}} }
Les messages transmis doivent h�riter de la classe
\verb|communication::message|. Chaque type de message doit avoir un code
identifiant son contenu (afin de faciliter le traitement par les destinataires).

\subsection{Organisation de la communication dans le jeu
            \index{communication!message \emph{(classe)}} }
Dans le jeu, il n'y a qu'un seul bureau de poste, repr�sent� par le singleton
\verb|ptb::post_office|
 \index{post office \emph{(classe ptb::)}}
 \index{bureau de poste}.

Les items souhaitant recevoir des messages doivent h�riter de
\verb|ptb::messageable_item|
 \index{messageable item \emph{(classe ptb::)}}
ou de \verb|ptb::renderable_messageable|
 \index{renderable messageable \emph{(classe ptb::)}}
.
