\documentclass{article}

%------------------------------------------------------------------------------
%Packages
\usepackage[francais]{babel}  
\usepackage[utf8]{inputenc}
\usepackage[T1]{fontenc}
\usepackage{graphicx}

%------------------------------------------------------------------------------

\title{Plee the Bear}
\author{Julien Jorge, Sébastien Angibaud}
\date{\today}

\begin{document}

%------------------------------------------------------------------------------

\maketitle
%\newpage
\tableofcontents
\newpage
%------------------------------------------------------------------------------

%------------------------------------------------------------------------------
\section{Avant propos}
Ces gens ont eu la merveilleuse idée de participer à l'élaboration de
ce document, et par conséquent aux bases même du jeu, en y apportant
leurs bonnes idées et leurs intéressants commentaires :

% garder l'ordre alphabétique des noms de famille dans cette liste
\begin{itemize}
\item Pierre Fradin ;
\item Julien Lassalle ;
\item Anthony Leroy ;
\item Florian Massuyeau ;
\item David Michineau.
\end{itemize}

%------------------------------------------------------------------------------
\section{Présentation du projet}
Plee the Bear est un jeu de plates-formes, libre, dans le style de ceux
réalisés dans les années 90. L'objectif est de réaliser un jeu libre
fun et original, dans la mesure du possible, vu que tout a déjà été
fait en matière de jeux de plates-formes. Pour donner un ordre d'idée
de ce que nous considérons comme fun, nos modèles sont les excellents
Mario, Sonic, Earthworm Jim, Ristar, et encore d'autres.

Le synopsis de l'histoire tient dans ces quelques lignes :
\begin{quote}
\emph{Seize heures et des poussières, Plee se réveille, fatigué. Il a encore
rêvé de cette époque formidable pendant laquelle il arpentait le monde
en compagnie de sa belle. Il plonge sa patte dans le pot de
miel. Vide. Tout les pots de la maison sont vides
d'ailleurs. <<~Encore un coup du gamin~>>, se dit-il. <<~Je m'en vais
te lui foutre un torgnole dont il se rappellera~>>.}

\emph{En suivant les traces de miel laissées sur le sol, Plee rejoint
l'orée de la forêt. Début du jeu.}
\end{quote}

Dans la suite de ce document seront présentés les éléments faisant le
gameplay, puis suivra la liste des environnements visités durant le
jeu.

%------------------------------------------------------------------------------
\section{Gameplay}

%- - - - - - - - - - - - - - - - - - - - - - - - - - - - - - - - - - - - - - - 
\subsection{Le héros}

Le héros du jeu, ou plutôt l'anti-héros, est un ours brun nommé Plee,
mesurant deux mètres quarante de haut et pesant une centaine de
kilos. Il est grognon et fainéant. Les caractéristiques ci-dessous
servent à sa progression dans le jeu.
\begin{itemize}
\item Plee a de bonnes griffes ;
\item il donne de bonnes baffes en marchant, ou à l'arrêt. Il fait un
      salto en courant ;
\item à l'arrêt, il fait des sauts plus ou moins haut suivant la puissance
      accumulée pour sauter. Un halo de lumière apparaît pour doser la
      puissance ;
\item il saute de mur en mur en s'agrippant avec ses griffes ;
\item il sait jeter des objets (pierres) ;
\item il peut creuser certains murs ;
\item il court vite et à quatre pattes ;
\item il se transforme en super ours blanc en dézippant sa
      fourrure. Il est invincible quand il est en ours blanc.
\end{itemize}

Plee est sensible à la chaleur, au froid, et ne sait pas respirer sous
l'eau. Dans un environnement d'eau, un indicateur permet de savoir à
quel point Plee a besoin d'air. De même, dans un environnement chaud,
son assèchement est indiqué et dans un environnement froid, un
indicateur permet de savoir si Plee est sur le point de geler.

%- - - - - - - - - - - - - - - - - - - - - - - - - - - - - - - - - - - - - - - 
\subsection{Dialogues, ambiance BD}
Les dialogues apparaissent sous la forme de bulles, comme dans une
bande dessinée, que ce soit les dialogues avec les personnages du jeu
ou les dialogues entre joueurs.

Des onomatopées sont affichées de temps en temps, quand Plee saute,
frappe, se cogne. Quand il tombe trop vite, une série de «~A~» de taille
décroissante, suivis d'un point d'exclamation, vont du haut de l'écran
vers Plee.

%- - - - - - - - - - - - - - - - - - - - - - - - - - - - - - - - - - - - - - - 
\subsection{Niveaux}
Le scénario emmène Plee dans de nombreux environnements. Chaque
environnement donne lieu à un niveau, séparé ensuite en deux actes.

Avant chaque niveau, un court texte explique succinctement la situation
au joueur, et son avancée dans le scénario.

Entre les deux actes d'un même niveau, durant le chargement du second,
nous faisons un clin d'\oe il à divers célèbres jeux au travers d'une
animation ou une image fixe de Plee déguisé en un personnage du dit
jeu.

En plus des niveau principaux, il y a des niveaux bonus accessibles
depuis divers endroits cachés dans le jeu.

%- - - - - - - - - - - - - - - - - - - - - - - - - - - - - - - - - - - - - - - 
\subsection{Équipement}

\subsubsection{Les pierres}
\label{sec:pierres}
Plee peut lancer des pierres. Elles éclatent lorsqu'elles atteignent
leur but et elles sont plus puissantes que les baffes. Une pierre se
brisant dans l'eau crée une petite bulle d'air que Plee peut aspirer.

\subsubsection{Les pouvoirs}
Trois types de pouvoirs peuvent être récupérés dans les
niveaux. Chaque pouvoir augmente les capacités de Plee et des pierres
sur leur domaine. Le pouvoir d'air permet de ne pas avoir à respirer
dans l'eau, le pouvoir du feu permet de ne pas se brûler et le pouvoir
d'eau permet d'éteindre certains éléments enflammés.

\subsubsection{Les bonus}
Des boîtes de bonus sont réparties dans les niveaux. On distingue les
boîtes de
\begin{itemize}
\item pierres en petite quantité : donnent une petite quantité de
      pierres à Plee ;
\item pierres en grande quantité : donnent une plus grande quantité de
      pierres à Plee ;
\item pouvoir (eau, air, feu) : donnent ledit pouvoir à Plee ;
\item petit Plee : donnent une vie supplémentaire à Plee ;
\item invincibilité : rendent Plee invincible pendant un court laps de
      temps.
\end{itemize}

%- - - - - - - - - - - - - - - - - - - - - - - - - - - - - - - - - - - - - - - 
\subsection{Ennemis}
À chaque environnement sont associés cinq ennemis. Trois d'entre eux
apparaissent dès le premier acte, les deux restant viennent s'ajouter
dans le second acte.

Certains ennemis sont immunisés contre certains pouvoirs et plus
sensibles à d'autres. Par exemple, un ennemi familier avec le feu sera
insensible aux pierres aux pouvoir de feu mais sera anéanti par une
pierre au pouvoir d'eau.

Lorsqu'un ennemi est tué, son âme s'élève. Plee peut (doit) prendre
cette âme pour récupérer de l'énergie.

%- - - - - - - - - - - - - - - - - - - - - - - - - - - - - - - - - - - - - - - 
\subsection{Comptage des points}
Chaque ennemi rapporte des points, qui sont d'autant plus importants
si les ennemis sont tués en série (il s'est passé moins de une seconde
depuis le dernier ennemi). Disons 100, 200, 500, 1~000, 10~000. à
partir de 10~000, un petit texte apparaît, disons <<~boucherie~>>,
<<~tuerie~>>, <<~double tuerie~>>, <<~carnage~>>.

À la fin de chaque niveau, un décompte de l'énergie restante du joueur
est effectué et converti en points. Si la traversé
d'un niveau doit se faire en un temps imparti, le joueur gagne
d'autant plus de points qu'il est allé vite. Le joueur gagne une vie
tous les 50~000 points. Le nombre d'ennemis est calculé et 100 points
sont attribués pour chaque. Un bonus de 50~000 est donné si tous les
ennemis du niveau ont été tués.

%- - - - - - - - - - - - - - - - - - - - - - - - - - - - - - - - - - - - - - - 
\subsection{Des astuces bien méritées}
Certains bonus ne trouvent leur intérêt que lorsque le jeu est
terminé. Ces bonus permettent d'obtenir quelques cheat codes. Chaque
bonus donnerait un code. Par exemple :
\begin{itemize}
\item sélection de niveau ;
\item invincibilité (ie. ne meurt jamais, sauf à tomber dans un
      trou) ;
\item énergie infinie.
\end{itemize}
La séquence de fin du jeu varie selon que le joueur ait terminé le jeu
en trichant ou pas.

%- - - - - - - - - - - - - - - - - - - - - - - - - - - - - - - - - - - - - - - 
\subsection{Sauvegardes et essais supplémentaires}

\subsubsection{Essais supplémentaires}
Chaque joueur a trois essais supplémentaires («~vies~») et un continue
de disponibles au début du jeu. Lorsqu'un joueur perd une vie, il est
repositionné au dernier point de sauvegarde rencontré dans l'acte, ou
au début de l'acte si aucun point de sauvegarde n'a été
rencontré. Ses pouvoirs sont supprimés et le compteur de vies est
décrémenté. Lorsqu'il n'a plus de vies, nous lui proposons d'utiliser
un continue. Le joueur sans vies reprend alors au début de l'acte en
cours. Le compteur d'essais est réinitialisé à trois et celui de
continues est décrémenté. S'il n'a plus de continues, c'est la fin du
jeu pour lui. Dans une partie à deux joueurs, l'autre joueur peut
continuer, quand il n'aura plus de continues le vrai grand écran
«~Game Over~» apparaîtra.

\subsubsection{Sauvegardes}
Nous stockons, à chaque fois qu'un acte est terminé :
\begin{itemize}
\item le nombre de joueurs (1 ou 2) ;
\item le niveau et sous niveau correspondant ;
\item le nombre de vies de chaque joueur ;
\item le nombre de continues de chaque joueur ;
\item le nombre de points de chaque joueur ;
\item les pouvoirs possédés ;
\item le nombre de bonus-cheat-codes trouvés.
\end{itemize}

L'écran permettant de choisir la sauvegarde à utiliser possède, pour
chaque sauvegarde, une petite image du niveau correspondant et toutes
les valeurs sauvées, sauf peut-être le nombre de bonus-cheat-codes
trouvés (car il s'agit d'une fonctionnalité cachée).

Le joueur peut choisir, pour une sauvegarde, le niveau auquel il veut
commencer (précédent ou égal à celui où s'est arrêtée sa
progression). Dans ce cas, la sauvegarde automatique ne sauve pas la
progression si l'acte terminé est avant celui qui est déjà dans la
sauvegarde. Par contre les vies et tous les autres paramètres sont
sauvés.

%- - - - - - - - - - - - - - - - - - - - - - - - - - - - - - - - - - - - - - - 
\subsection{Spécificités du jeu à deux joueurs}
\subsubsection{La caméra et le mode multi-joueurs}
La caméra se centre entre les deux joueurs. Si un joueur sort de
l'écran, la caméra se centre automatiquement sur l'un des deux,
aléatoirement ou alternativement. Il est possible de forcer la caméra
sur l'un des deux joueurs, ce qui laisse l'autre à l'abandon. Plus
précisément, chaque joueur a la possibilité de placer la caméra sur
lui.

%------------------------------------------------------------------------------
\section{Les niveaux}

En remplacement aux déguisements proposés dans
les prochaines pages, les personnages suivants peuvent être utilisés :
\begin{itemize}
\item un Lemming ;
\item Jim, de Earthworm Jim ;
\item Bomberman ;
\item un canard de Duck Hunt ;
\item Kong, de Donkey Konk Country ;
\item un personnage de Street Of Rage ;
\item Duke, de Duke Nukem ;
\item une tortue ninja.
\end{itemize}

%- - - - - - - - - - - - - - - - - - - - - - - - - - - - - - - - - - - - - - - 
\subsection{La forêt}
\subsubsection{Environnement}
<<~C'est loin d'être le paradis sur Terre. Il fait froid, sombre et
humide. Derrière chaque arbre se cache une sale bestiole prête à vous
bondir dessus. D'abord ces saletés de guêpes qui vous bombardent à vue
; un jet de dards cadencé tel une mitraillette. Ensuite viendront les
pics verts qui piquent sur vous si vous vous approchez de trop
près. Pour peu que vous soyez un peu dans un mauvais jour, vous
tomberez certainement nez à nez avec une laie et ses marcassins. Ces
petits excités feront tout pour vous faire tomber.~>>

\subsubsection{Le boss}
Un castor géant, avec des dents qui vont jusqu'au sol. Ses dents lui
servent de bouclier, il fait trembler le sol en le tapant avec sa
queue. Il saute d'un côté à l'autre de la pièce et donne un coup de
queue latéral quand on s'approche de trop près.

\subsubsection{Transition entre les parties}
Déguisement de Belmont, le type de Castlevania.

\subsubsection{Transition avec le niveau suivant}
Après avoir vaincu le boss, Plee voit son fils entrer dans un château.

%- - - - - - - - - - - - - - - - - - - - - - - - - - - - - - - - - - - - - - - 
\subsection{Le ch\^ateau}
\subsubsection{Environnement}
<<~Ce château a tout l'air d'être abandonné. C'est humide et plein de
poussière ; tout juste bon à être démoli. Autrefois occupé par un
couple de magiciens amateurs, on n'y a pas vu quiconque depuis
quelques siècles. Comme tout château abandonné, il y a une légende
derrière. Pour celui-ci on raconte que, par une nuit d'orage, les
propriétaires se seraient disputés. Quelques semaines plus tard ils se
séparèrent et aucun ne voulu récupérer le château. Depuis, il est à
l'abandon. On dit que le diable aurait profité de l'occasion pour s'y
installer. Il surveillerait les lieux à l'aide d'une armée de
squelettes, d'âmes déchues et d'autres trucs à trouver. Enfin bon, avec moi
les squelettes ne feront pas de vieux os...~>>

\subsubsection{Ennemis}
Pour les monstres, nous pourrons avoir, par exemple, des squelettes,
des spectres (fantômes), des morts-vivants genre goules, des
gargouilles.

Les fantômes ne sont pas des ennemis tuables. Il y aura deux types de
fantômes ; ceux qui s'agglutinent autour de Plee pour gêner la
visibilité et ceux qui prennent possession de Plee. Pour ces derniers,
il y aura encore deux catégories. Certains changeront juste la
configuration des touches, d'autres feront faire n'importe quoi à
Plee.

\subsubsection{Secrets}
Il y aura dans une partie du château un fantôme isolé. Si le joueur se
laisse posséder, il aura la possibilité de passer à travers un mur
pour accéder à une pièce secrète.

\subsubsection{Le boss}
Un tigre enflammé. C'est une épreuve d'endurance. Le tigre brûle à
petit feu et fini par mourir. Il est triste de brûler, donc il pleure.
Le challenge consiste à éviter les petites boules de feu qui émanent
de lui. Ses larmes s'évaporent et créent un nuage, qui émet des
éclairs (dangereux pour Plee). Le tigre meurt, la pluie tombe et
l'éteint. L'idéal serait de voir le niveau de l'eau monter petit à
petit. Plee pourrait monter sur des décors pour échapper à la montée
des eaux (genre des objets flottants).

\subsubsection{Transition entre les parties}
Déguisement de Mario.

\subsubsection{Transition avec le niveau suivant}

%- - - - - - - - - - - - - - - - - - - - - - - - - - - - - - - - - - - - - - - 
\subsection{Le centre de la Terre}
\subsubsection{Environnement}
<<~Je m'rappelle que, quand j'étais gamin, papa m'amenait à la
mine. Il m'amenait le matin et revenait me chercher le soir. Pendant
que je trimais à briser des pierres, ce petit malin s'en allait faire
le tour des bars. Heureusement j'y ai rencontré d'autres gamins dans
la même situation ; on s'est serré les coudes tous les trois. Il y
avait Bobby, qui s'est retrouvé coincé sous un éboulement, le pauvre
était au mauvais endroit au mauvais moment, et il y avait Jimmy. Ce
petit garnement passait son temps à provoquer des éboulements à grand
coups de dynamite. Un jour qu'il alluma la mèche, Bobby, en contrebas,
lui cria quelque chose. Il n'eut pas le temps de réagir que la bombe
avait déjà explosé. Quels idiots ces deux là. Le pire c'est que
personne d'autre ne savait o\`u étaient placées les autre charges. Du
coup ça devenait dangereux de rester là, enfin je veux dire que ça
devenait \textbf{encore} plus dangereux... J'ai pas attendu qu'ils se
décident, je me suis barré. Ils ont fermé la mine une semaine plus
tard.~>>

Le niveau se fait globalement dans le sens de la descente. Il pourrait
y avoir des endroits totalement dans la pénombre. Le niveau pourrait
être une succession de grottes et de tunnels. Ça commencerait dans un
environnement plutôt humide, aux teintes noires/bleues, avec des
flaques, et terminerait dans un environnement chaud, aux teintes
noires/rouges, avec des flaques de lave. Si la technique nous le
permet, un parcourt en chariot serait le bienvenu.

\subsubsection{Ennemis}
En ce qui concerne les ennemis, on va trouver des chauves souris, des
petits diablotins (pour garder le lien avec le diable par rapport au
niveau précédent), des multivers et des taupes.

\subsubsection{Le boss}
La pièce possède des plate-formes en hauteur sur lesquelles le diable
et Plee peuvent monter.  Le diable nous met des coups de queue si on
est trop prêt et nous crache des boules de feu <<~à tête chercheuse~>>
lorsqu'on est loin.  En retombant des plate-formes, le diable provoque
un petit éboulement. Plee peut ramasser certaines des pierres
tombées. Le but est d'en envoyer dix-huit sur le diable (avec une
métamorphose du diable toutes les 6 pierres). Si Plee a le pouvoir de
feu, les pierres augmentent l'énergie du diable (il faut donc se
débarrasser du pouvoir). Le diable peut faire jaillir de la lave du
sol, avec un petit bout de sol. Plee peut monter sur ce bout de sol
pour monter sur les plate-formes en hauteur.

\subsubsection{Transition entre les parties}
Déguisement de Lara Croft.

\subsubsection{Transition avec le niveau suivant}

%- - - - - - - - - - - - - - - - - - - - - - - - - - - - - - - - - - - - - - - 
\subsection{La montagne}
\subsubsection{Environnement}
<<~J'ai froid. Voilà encore un coin dans lequel j'aimerais ne pas
m'attarder. D'abord parce que j'ai froid, ensuite parce que c'est une
des régions les plus hostiles que je connaisse. On parle d'un monstre
qui hanterait les cimes enneigées et qui transformerait en simple
casse-croûte quiconque s'en approcherait. Si vous voulez mon avis
c'est encore une de ces histoires inventée par des parents fatigués
pour obliger leurs gamins à manger toute sorte de nourriture
décevante. Ceci dit, ce dont je suis s\^ur, c'est que l'on y croise sans
problème des dahuts, des saint-bernards titubants ou encore des lamas
qui vous crachent au visage (ils tirent loin ces idiots). Ça plus
les crevasses et les stalactites qui vous tombent sur le coin du
museau, je vous le dit, c'est loin d'être une partie de plaisir.~>>

L'idée est d'avoir une progression globalement verticale. Le niveau
commencerait dans une ambiance un peu rocheuse, puis se terminerait
dans un environnement enneigé. On passe par des grottes aux parois
lisses qui reflètent joueur. On pourrait même commencer le niveau dans
une mine, pour faire la transition avec le niveau précédent ; quoi que
du coup le niveau précédent commencerait et se terminerait dans une
mine.

\subsubsection{Le boss}
Dans une grotte bien éclairée qui reflète vos moindres gestes et
où les stalactites tombent toutes les dix secondes, un yéti a 
une fâcheuse tendance à vous agresser.

\subsubsection{Transition entre les parties}
Déguisement de Link, le héros des jeux Zelda.

\subsubsection{Transition avec le niveau suivant}

%- - - - - - - - - - - - - - - - - - - - - - - - - - - - - - - - - - - - - - - 
\subsection{Le port}
\subsubsection{Environnement}
<<~Fichu port. Il faut passer de quais en bateaux tout en évitant les
boulets de canons. Du haut des mâts, les hordes de mouettes vous
attaquent. En fond de cales,  (un autre problème). De vieilles grues
de chargement permettent de prendre de l'altitude, mais gare à celles
qui cèdent et vous écrasent. Dans les bateaux les plus délabrés, il va
falloir se mettre à l'eau. Cependant si les bateaux vous font peur,
vous pouvez toujours tenter de vous frayer un passage via les
égouts.~>>

La première moitié du niveau se passe sur les quais, la seconde en
pleine mer, sur des bateaux en file indienne (la progression devient
très horizontale).

\subsubsection{Le boss}
Sur le pont d'un bateau, il s'agit de se battre contre un squelette
fantôme pirate, son épée et ses canons. Une fois le boss tué, le
bateau coule.

\subsubsection{Transition entre les parties}
Déguisement de Ken, de Street Fighter.

\subsubsection{Transition avec le niveau suivant}

%- - - - - - - - - - - - - - - - - - - - - - - - - - - - - - - - - - - - - - - 
\subsection{L'océan}
\subsubsection{Environnement}
<<~Dur, dur de respirer. Il faut jeter des pierres pour récupérer un
peu d'air (cf. \ref{sec:pierres}). Coté danger, il y a des sèches qui
jettent de l'encre, des gymnotes qui électrocutent tout ce qu'elles
touchent, sans parler des courants marins vous emportant vers des
requins affamés.~>>

\subsubsection{Le boss}
Une pieuvre immense avec ses huit tentacules. Il faut couper toutes
les tentacules pour la vaincre.

\subsubsection{Transition entre les parties}
Déguisement d'Ecco, le dauphin.

\subsubsection{Transition avec le niveau suivant}

%- - - - - - - - - - - - - - - - - - - - - - - - - - - - - - - - - - - - - - - 
\subsection{L'Égypte}
\subsubsection{Environnement}
<<~J'ai chaud. Si vous voulez trouver quelques points d'eau, il faudra
visiter quelques tombeaux. Si vous visitez un tombeau, vous n'êtes pas
sûr d'en ressortir. Ce sont de dangereux labyrinthes. Des sables
mouvants, Des chameaux, des momies idiotes qui vous fouettent avec
leurs bandes.~>>

\subsubsection{Le boss}
Un sphinx avec un bon souffle et de bonnes griffes.

\subsubsection{Transition entre les parties}
Déguisement de Sonic.

\subsubsection{Transition avec le niveau suivant}

%- - - - - - - - - - - - - - - - - - - - - - - - - - - - - - - - - - - - - - - 
\subsection{Le ciel}
\subsubsection{Environnement}
Cette partie du jeu se déroule en montgolfière. Genre de niveau
«~aérien~» comme il en existe dans pas mal de jeux. Plee donne ses
coups à partir de la montgolfière. Les ennemis pourraient être des
oiseaux au regard méchant, des nuages d'orage. On peut imaginer qu'il
faille récupérer des bonbonnes de gaz pour que la montgolfière
continue de voler.

\subsubsection{Le boss}
Dieu qui lance des éclairs, des anges déchus (il arrache les ailes des
anges qui passent dans le coin). Les anges déchus se transforment en
diablotins. Les diablotins attaquent Dieu, le joueur doit envoyer des
pierres sur Dieu pour qu'il ne puisse pas parer les attaques des
diablotins.

\subsubsection{Transition entre les parties}
Déguisement de Joe Musashi, le héros des jeux Shinobi.

\subsubsection{Transition avec le niveau suivant}
Pas de transition, c'est la fin du jeu ;

%------------------------------------------------------------------------------
\section{Niveaux bonus}
\subsection{Petit niveau}
Avec plein de miel. Ouverture cachée dans le niveau.

%- - - - - - - - - - - - - - - - - - - - - - - - - - - - - - - - - - - - - - - 
\subsection{La lune, le niveau caché}
\subsubsection{Environnement}
La lune est un niveau caché dont l'entrée sera reste à placer
dans l'un des autres niveaux. Parmi les possibilités envisagée pour
accéder à ce niveau, nous avons :
\begin{itemize}
\item terminer un niveau avec nombre de boules d'énergie n'ayant
qu'un seul chiffre (ex.~333, 44) et supérieur à 10. On peut
éventuellement ajouter la nécessité d'avoir le même état sur l'horloge
(ex.~1~min. 11~sec., 5~min. 55~sec.) ;
\item terminer l'avant dernier niveau en ayant découvert tous les
passages secrets des niveaux précédents.
\end{itemize}

Dans ce niveau, la gravité est moindre et le décor assez épuré.

\subsubsection{Ennemis}
Pour les ennemis on peut envisager les difficultés suivantes:
\begin{itemize}
\item pluies de météorites ;
\item soit des recharges d'air à trouver (bonus) qui permettent à Plee
      de remplir une réserve d'air qui s'épuise au court du niveau ;
\item soit un temps limité pour réaliser le niveau sous peine
      d'asphyxie et donc de mort.
\end{itemize}

\subsubsection{Le boss}
Aucun pour l'instant.

\subsubsection{Transition entre les parties}
Déguisement du héros de Doom (rendu fa\c con premier Doom).

\subsubsection{Transition avec le niveau suivant}

\end{document}
